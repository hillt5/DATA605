% Options for packages loaded elsewhere
\PassOptionsToPackage{unicode}{hyperref}
\PassOptionsToPackage{hyphens}{url}
%
\documentclass[
]{article}
\usepackage{lmodern}
\usepackage{amssymb,amsmath}
\usepackage{ifxetex,ifluatex}
\ifnum 0\ifxetex 1\fi\ifluatex 1\fi=0 % if pdftex
  \usepackage[T1]{fontenc}
  \usepackage[utf8]{inputenc}
  \usepackage{textcomp} % provide euro and other symbols
\else % if luatex or xetex
  \usepackage{unicode-math}
  \defaultfontfeatures{Scale=MatchLowercase}
  \defaultfontfeatures[\rmfamily]{Ligatures=TeX,Scale=1}
\fi
% Use upquote if available, for straight quotes in verbatim environments
\IfFileExists{upquote.sty}{\usepackage{upquote}}{}
\IfFileExists{microtype.sty}{% use microtype if available
  \usepackage[]{microtype}
  \UseMicrotypeSet[protrusion]{basicmath} % disable protrusion for tt fonts
}{}
\makeatletter
\@ifundefined{KOMAClassName}{% if non-KOMA class
  \IfFileExists{parskip.sty}{%
    \usepackage{parskip}
  }{% else
    \setlength{\parindent}{0pt}
    \setlength{\parskip}{6pt plus 2pt minus 1pt}}
}{% if KOMA class
  \KOMAoptions{parskip=half}}
\makeatother
\usepackage{xcolor}
\IfFileExists{xurl.sty}{\usepackage{xurl}}{} % add URL line breaks if available
\IfFileExists{bookmark.sty}{\usepackage{bookmark}}{\usepackage{hyperref}}
\hypersetup{
  pdftitle={Assignment 1 for DATA 605},
  pdfauthor={Thomas Hill},
  hidelinks,
  pdfcreator={LaTeX via pandoc}}
\urlstyle{same} % disable monospaced font for URLs
\usepackage[margin=1in]{geometry}
\usepackage{color}
\usepackage{fancyvrb}
\newcommand{\VerbBar}{|}
\newcommand{\VERB}{\Verb[commandchars=\\\{\}]}
\DefineVerbatimEnvironment{Highlighting}{Verbatim}{commandchars=\\\{\}}
% Add ',fontsize=\small' for more characters per line
\usepackage{framed}
\definecolor{shadecolor}{RGB}{248,248,248}
\newenvironment{Shaded}{\begin{snugshade}}{\end{snugshade}}
\newcommand{\AlertTok}[1]{\textcolor[rgb]{0.94,0.16,0.16}{#1}}
\newcommand{\AnnotationTok}[1]{\textcolor[rgb]{0.56,0.35,0.01}{\textbf{\textit{#1}}}}
\newcommand{\AttributeTok}[1]{\textcolor[rgb]{0.77,0.63,0.00}{#1}}
\newcommand{\BaseNTok}[1]{\textcolor[rgb]{0.00,0.00,0.81}{#1}}
\newcommand{\BuiltInTok}[1]{#1}
\newcommand{\CharTok}[1]{\textcolor[rgb]{0.31,0.60,0.02}{#1}}
\newcommand{\CommentTok}[1]{\textcolor[rgb]{0.56,0.35,0.01}{\textit{#1}}}
\newcommand{\CommentVarTok}[1]{\textcolor[rgb]{0.56,0.35,0.01}{\textbf{\textit{#1}}}}
\newcommand{\ConstantTok}[1]{\textcolor[rgb]{0.00,0.00,0.00}{#1}}
\newcommand{\ControlFlowTok}[1]{\textcolor[rgb]{0.13,0.29,0.53}{\textbf{#1}}}
\newcommand{\DataTypeTok}[1]{\textcolor[rgb]{0.13,0.29,0.53}{#1}}
\newcommand{\DecValTok}[1]{\textcolor[rgb]{0.00,0.00,0.81}{#1}}
\newcommand{\DocumentationTok}[1]{\textcolor[rgb]{0.56,0.35,0.01}{\textbf{\textit{#1}}}}
\newcommand{\ErrorTok}[1]{\textcolor[rgb]{0.64,0.00,0.00}{\textbf{#1}}}
\newcommand{\ExtensionTok}[1]{#1}
\newcommand{\FloatTok}[1]{\textcolor[rgb]{0.00,0.00,0.81}{#1}}
\newcommand{\FunctionTok}[1]{\textcolor[rgb]{0.00,0.00,0.00}{#1}}
\newcommand{\ImportTok}[1]{#1}
\newcommand{\InformationTok}[1]{\textcolor[rgb]{0.56,0.35,0.01}{\textbf{\textit{#1}}}}
\newcommand{\KeywordTok}[1]{\textcolor[rgb]{0.13,0.29,0.53}{\textbf{#1}}}
\newcommand{\NormalTok}[1]{#1}
\newcommand{\OperatorTok}[1]{\textcolor[rgb]{0.81,0.36,0.00}{\textbf{#1}}}
\newcommand{\OtherTok}[1]{\textcolor[rgb]{0.56,0.35,0.01}{#1}}
\newcommand{\PreprocessorTok}[1]{\textcolor[rgb]{0.56,0.35,0.01}{\textit{#1}}}
\newcommand{\RegionMarkerTok}[1]{#1}
\newcommand{\SpecialCharTok}[1]{\textcolor[rgb]{0.00,0.00,0.00}{#1}}
\newcommand{\SpecialStringTok}[1]{\textcolor[rgb]{0.31,0.60,0.02}{#1}}
\newcommand{\StringTok}[1]{\textcolor[rgb]{0.31,0.60,0.02}{#1}}
\newcommand{\VariableTok}[1]{\textcolor[rgb]{0.00,0.00,0.00}{#1}}
\newcommand{\VerbatimStringTok}[1]{\textcolor[rgb]{0.31,0.60,0.02}{#1}}
\newcommand{\WarningTok}[1]{\textcolor[rgb]{0.56,0.35,0.01}{\textbf{\textit{#1}}}}
\usepackage{graphicx}
\makeatletter
\def\maxwidth{\ifdim\Gin@nat@width>\linewidth\linewidth\else\Gin@nat@width\fi}
\def\maxheight{\ifdim\Gin@nat@height>\textheight\textheight\else\Gin@nat@height\fi}
\makeatother
% Scale images if necessary, so that they will not overflow the page
% margins by default, and it is still possible to overwrite the defaults
% using explicit options in \includegraphics[width, height, ...]{}
\setkeys{Gin}{width=\maxwidth,height=\maxheight,keepaspectratio}
% Set default figure placement to htbp
\makeatletter
\def\fps@figure{htbp}
\makeatother
\setlength{\emergencystretch}{3em} % prevent overfull lines
\providecommand{\tightlist}{%
  \setlength{\itemsep}{0pt}\setlength{\parskip}{0pt}}
\setcounter{secnumdepth}{-\maxdimen} % remove section numbering
\ifluatex
  \usepackage{selnolig}  % disable illegal ligatures
\fi

\title{Assignment 1 for DATA 605}
\author{Thomas Hill}
\date{}

\begin{document}
\maketitle

ASSIGNMENT 1 IS 605 FUNDAMENTALS OF COMPUTATIONAL MATHEMATICS - FALL
2014

\begin{enumerate}
\def\labelenumi{\arabic{enumi}.}
\tightlist
\item
  Problem set 1
\end{enumerate}

You can think of vectors representing many dimensions of related
information. For instance, Netflix might store all the ratings a user
gives to movies in a vector. This is clearly a vector of very large
dimensions (in the millions) and very sparse as the user might have
rated only a few movies. Similarly, Amazon might store the items
purchased by a user in a vector, with each slot or dimension
representing a unique product and the value of the slot, the number of
such items the user bought. One task that is frequently done in these
settings is to find similarities between users. And, we can use
dot-product between vectors to do just that. As you know, the
dot-product is proportional to the length of two vectors and to the
angle between them. In fact, the dot-product between two vectors,
normalized by their lengths is called as the cosine distance and is
frequently used in recommendation engines.

\begin{enumerate}
\def\labelenumi{(\arabic{enumi})}
\tightlist
\item
  Calculate the dot product u:v where u = {[}0:5; 0:5{]} and v =
  {[}3;􀀀4{]}
\end{enumerate}

\begin{Shaded}
\begin{Highlighting}[]
\NormalTok{u \textless{}{-}}\StringTok{ }\KeywordTok{matrix}\NormalTok{(}\KeywordTok{c}\NormalTok{(}\FloatTok{0.5}\NormalTok{,}\FloatTok{0.5}\NormalTok{), }\DecValTok{1}\NormalTok{,}\DecValTok{2}\NormalTok{)}
\NormalTok{v \textless{}{-}}\StringTok{ }\KeywordTok{matrix}\NormalTok{(}\KeywordTok{c}\NormalTok{(}\DecValTok{3}\NormalTok{,}\DecValTok{4}\NormalTok{),}\DecValTok{1}\NormalTok{,}\DecValTok{2}\NormalTok{)}

\NormalTok{u\_v\_dot\_m \textless{}{-}}\StringTok{ }\KeywordTok{matrix}\NormalTok{(}\KeywordTok{c}\NormalTok{(u[}\DecValTok{0}\NormalTok{]}\OperatorTok{*}\NormalTok{v[}\DecValTok{0}\NormalTok{],u[}\DecValTok{1}\NormalTok{]}\OperatorTok{*}\NormalTok{v[}\DecValTok{1}\NormalTok{]),}\DecValTok{1}\NormalTok{,}\DecValTok{2}\NormalTok{)}

\NormalTok{u\_v\_dot\_product \textless{}{-}}\StringTok{ }\NormalTok{(u\_v\_dot\_m[}\DecValTok{1}\NormalTok{,}\DecValTok{1}\NormalTok{]}\OperatorTok{**}\DecValTok{2} \OperatorTok{+}\StringTok{ }\NormalTok{u\_v\_dot\_m[}\DecValTok{1}\NormalTok{,}\DecValTok{2}\NormalTok{]}\OperatorTok{**}\DecValTok{2}\NormalTok{)}\OperatorTok{**}\FloatTok{0.5}

\KeywordTok{print}\NormalTok{(u\_v\_dot\_m) }\CommentTok{\#vector}
\end{Highlighting}
\end{Shaded}

\begin{verbatim}
##      [,1] [,2]
## [1,]  1.5  1.5
\end{verbatim}

\begin{Shaded}
\begin{Highlighting}[]
\KeywordTok{print}\NormalTok{(}\KeywordTok{round}\NormalTok{(u\_v\_dot\_product,}\DecValTok{3}\NormalTok{)) }\CommentTok{\#length of vector}
\end{Highlighting}
\end{Shaded}

\begin{verbatim}
## [1] 2.121
\end{verbatim}

\textbf{The dot product is a special case of the inner product of a
matrix, and is calculated by finding the product of the \emph{ith}
element in two vectors of the same dimension, then using the distance
formula to determine a scalar value.}

\begin{enumerate}
\def\labelenumi{(\arabic{enumi})}
\setcounter{enumi}{1}
\tightlist
\item
  What are the lengths of u and v? Please note that the mathematical
  notion of the length of a vector is not the same as a computer science
  definition.
\end{enumerate}

\begin{Shaded}
\begin{Highlighting}[]
\NormalTok{u\_len \textless{}{-}}\StringTok{ }\NormalTok{(u[}\DecValTok{1}\NormalTok{,}\DecValTok{1}\NormalTok{]}\OperatorTok{**}\DecValTok{2} \OperatorTok{+}\StringTok{ }\NormalTok{u[}\DecValTok{1}\NormalTok{,}\DecValTok{2}\NormalTok{]}\OperatorTok{**}\DecValTok{2}\NormalTok{)}\OperatorTok{**}\FloatTok{0.5}
\NormalTok{v\_len \textless{}{-}}\StringTok{ }\NormalTok{(v[}\DecValTok{1}\NormalTok{,}\DecValTok{1}\NormalTok{]}\OperatorTok{**}\DecValTok{2} \OperatorTok{+}\StringTok{ }\NormalTok{v[}\DecValTok{1}\NormalTok{,}\DecValTok{2}\NormalTok{]}\OperatorTok{**}\DecValTok{2}\NormalTok{)}\OperatorTok{**}\FloatTok{0.5}

\KeywordTok{print}\NormalTok{(}\KeywordTok{round}\NormalTok{(u\_len,}\DecValTok{3}\NormalTok{))}
\end{Highlighting}
\end{Shaded}

\begin{verbatim}
## [1] 0.707
\end{verbatim}

\begin{Shaded}
\begin{Highlighting}[]
\KeywordTok{print}\NormalTok{(v\_len)}
\end{Highlighting}
\end{Shaded}

\begin{verbatim}
## [1] 5
\end{verbatim}

\begin{enumerate}
\def\labelenumi{(\arabic{enumi})}
\setcounter{enumi}{2}
\tightlist
\item
  What is the linear combination: 3u -􀀀 2v?
\end{enumerate}

\begin{Shaded}
\begin{Highlighting}[]
\NormalTok{u\_v\_comb \textless{}{-}}\StringTok{ }\DecValTok{3} \OperatorTok{*}\StringTok{ }\NormalTok{u }\OperatorTok{{-}}\StringTok{ }\DecValTok{2} \OperatorTok{*}\StringTok{ }\NormalTok{v}

\KeywordTok{print}\NormalTok{(u\_v\_comb)}
\end{Highlighting}
\end{Shaded}

\begin{verbatim}
##      [,1] [,2]
## [1,] -4.5 -6.5
\end{verbatim}

\begin{enumerate}
\def\labelenumi{(\arabic{enumi})}
\setcounter{enumi}{3}
\tightlist
\item
  What is the angle between u and v?
\end{enumerate}

\begin{Shaded}
\begin{Highlighting}[]
\NormalTok{u\_v\_angle \textless{}{-}}\StringTok{ }\KeywordTok{acos}\NormalTok{(u\_v\_dot\_product }\OperatorTok{/}\StringTok{ }\NormalTok{(u\_len }\OperatorTok{*}\StringTok{ }\NormalTok{v\_len)) }\CommentTok{\#this is in radians}

\NormalTok{u\_v\_angle\_deg \textless{}{-}}\StringTok{ }\NormalTok{u\_v\_angle }\OperatorTok{*}\StringTok{ }\NormalTok{(}\DecValTok{180}\OperatorTok{/}\StringTok{ }\NormalTok{pi)}

\KeywordTok{print}\NormalTok{(}\KeywordTok{round}\NormalTok{(u\_v\_angle\_deg,}\DecValTok{3}\NormalTok{))}
\end{Highlighting}
\end{Shaded}

\begin{verbatim}
## [1] 53.13
\end{verbatim}

\textbf{The angle between these two vectors is approximately 53 degrees.
I found this by using the formula for dot product of two vectors, which
is equal to the product of lengths a and b multiplied by the cosine of
their angle.}

You can use R-markdown to submit your responses to this problem set. If
you decide to do it in paper, then please either scan it or take a
picture using a smartphone and attach that picture. Please make sure
that the picture is legible before submitting.

\begin{enumerate}
\def\labelenumi{\arabic{enumi}.}
\setcounter{enumi}{1}
\tightlist
\item
  Problem set 2 Set up a system of equations with 3 variables and 3
  constraints and solve for x. Please write a function in R that will
  take two variables (matrix A \& constraint vector b) and solve using
  elimination. Your function should produce the right answer for the
  system of equations for any 3-variable, 3-equation system. You don't
  have to worry about degenerate cases and can safely assume that the
  function will only be tested with a system of equations that has a
  solution. Please note that you do have to worry about zero pivots,
  though. Please note that you should not use the built-in function
  solve to solve this system or use matrix inverses. The approach that
  you should employ is to construct an Upper Triangular Matrix and then
  back-substitute to get the solution. Alternatively, you can augment
  the matrix A with vector b and jointly apply the Gauss Jordan
  elimination procedure.
\end{enumerate}

IS 605 FUNDAMENTALS OF COMPUTATIONAL MATHEMATICS - FALL 2014 Please test
it with the system below and it should produce a solution x =
{[}􀀀1:55;􀀀0:32; 0:95{]}

\begin{Shaded}
\begin{Highlighting}[]
\NormalTok{getSolutions \textless{}{-}}\StringTok{ }\ControlFlowTok{function}\NormalTok{(matA, matB) \{}

\NormalTok{  matAB \textless{}{-}}\StringTok{ }\KeywordTok{cbind}\NormalTok{(matA,matB) }\CommentTok{\#combination of coefficient and constraint matrices}
  \ControlFlowTok{if}\NormalTok{(matAB[}\DecValTok{1}\NormalTok{,}\DecValTok{1}\NormalTok{] }\OperatorTok{!=}\StringTok{ }\DecValTok{0}\NormalTok{) \{ }
\NormalTok{    matAB[}\DecValTok{1}\NormalTok{,] \textless{}{-}}\StringTok{ }\NormalTok{matAB[}\DecValTok{1}\NormalTok{,]}\OperatorTok{/}\NormalTok{matAB[}\DecValTok{1}\NormalTok{,}\DecValTok{1}\NormalTok{] }\CommentTok{\#set the first row\textquotesingle{}s first column equal to one}
\NormalTok{  \}}
  \ControlFlowTok{if}\NormalTok{(matAB[}\DecValTok{2}\NormalTok{,}\DecValTok{2}\NormalTok{] }\OperatorTok{!=}\StringTok{ }\DecValTok{0}\NormalTok{) \{}
\NormalTok{    matAB[}\DecValTok{2}\NormalTok{,] \textless{}{-}}\StringTok{ }\NormalTok{matAB[}\DecValTok{2}\NormalTok{,] }\OperatorTok{{-}}\StringTok{ }\NormalTok{(matAB[}\DecValTok{1}\NormalTok{,] }\OperatorTok{*}\NormalTok{matAB[}\DecValTok{2}\NormalTok{,}\DecValTok{1}\NormalTok{]) }\CommentTok{\#set the second row\textquotesingle{}s first column equal to zero}
\NormalTok{    matAB[}\DecValTok{2}\NormalTok{,] \textless{}{-}}\StringTok{ }\NormalTok{matAB[}\DecValTok{2}\NormalTok{,]}\OperatorTok{/}\NormalTok{matAB[}\DecValTok{2}\NormalTok{,}\DecValTok{2}\NormalTok{] }\CommentTok{\#set the second row\textquotesingle{}s second column equal to one}
\NormalTok{  \}}
  
  \ControlFlowTok{if}\NormalTok{(matAB[}\DecValTok{3}\NormalTok{,}\DecValTok{3}\NormalTok{] }\OperatorTok{!=}\StringTok{ }\DecValTok{0}\NormalTok{) \{}
\NormalTok{    matAB[}\DecValTok{3}\NormalTok{,] \textless{}{-}}\StringTok{ }\NormalTok{matAB[}\DecValTok{3}\NormalTok{,] }\OperatorTok{{-}}\StringTok{ }\NormalTok{(matAB[}\DecValTok{1}\NormalTok{,] }\OperatorTok{*}\NormalTok{matAB[}\DecValTok{3}\NormalTok{,}\DecValTok{1}\NormalTok{]) }\CommentTok{\#set the third row\textquotesingle{}s first column equal to zero}
\NormalTok{    matAB[}\DecValTok{3}\NormalTok{,] \textless{}{-}}\StringTok{ }\NormalTok{matAB[}\DecValTok{3}\NormalTok{,] }\OperatorTok{{-}}\StringTok{ }\NormalTok{(matAB[}\DecValTok{2}\NormalTok{,] }\OperatorTok{*}\NormalTok{matAB[}\DecValTok{3}\NormalTok{,}\DecValTok{2}\NormalTok{]) }\CommentTok{\#set the third row\textquotesingle{}s first column equal to zero}
\NormalTok{    matAB[}\DecValTok{3}\NormalTok{,] \textless{}{-}}\StringTok{ }\NormalTok{matAB[}\DecValTok{3}\NormalTok{,]}\OperatorTok{/}\NormalTok{matAB[}\DecValTok{3}\NormalTok{,}\DecValTok{3}\NormalTok{] }\CommentTok{\#set the third row\textquotesingle{}s third column equal to one}
\NormalTok{  \}}
  
\NormalTok{  matSols \textless{}{-}}\StringTok{ }\NormalTok{matAB[,}\DecValTok{4}\NormalTok{] }\CommentTok{\#define new matrix containing the solutions}
\NormalTok{  matSols[}\DecValTok{2}\NormalTok{] =}\StringTok{ }\NormalTok{matSols[}\DecValTok{2}\NormalTok{] }\OperatorTok{{-}}\StringTok{ }\NormalTok{matAB[}\DecValTok{2}\NormalTok{,}\DecValTok{3}\NormalTok{] }\OperatorTok{*}\StringTok{ }\NormalTok{matSols[}\DecValTok{3}\NormalTok{] }\CommentTok{\#subsitute the solution of x3 into row 2}
\NormalTok{  matSols[}\DecValTok{1}\NormalTok{] =}\StringTok{ }\NormalTok{matSols[}\DecValTok{1}\NormalTok{] }\OperatorTok{{-}}\StringTok{ }\NormalTok{matAB[}\DecValTok{1}\NormalTok{,}\DecValTok{3}\NormalTok{] }\OperatorTok{*}\StringTok{ }\NormalTok{matSols[}\DecValTok{3}\NormalTok{] }\OperatorTok{{-}}\StringTok{ }\NormalTok{matAB[}\DecValTok{1}\NormalTok{,}\DecValTok{2}\NormalTok{] }\OperatorTok{*}\StringTok{ }\NormalTok{matSols[}\DecValTok{2}\NormalTok{] }\CommentTok{\#substitute the solutions of x2, x3 into row 1}
\NormalTok{  matSols \textless{}{-}}\StringTok{ }\KeywordTok{round}\NormalTok{(matSols,}\DecValTok{2}\NormalTok{)}

  \KeywordTok{print}\NormalTok{(matSols)}
\NormalTok{\}}
\end{Highlighting}
\end{Shaded}

\textbf{The above function accepts two matrices: the coefficients and
constraints, then generates an upper triangular matrix by eliminating
the affected columns from rows 2 and 3. It also checks whether these
leading functions are zero to begin with. Finally, it substitutes the
known solutions into the first two rows and provides an an array of
solutions.}

\begin{Shaded}
\begin{Highlighting}[]
\NormalTok{assign\_matrix \textless{}{-}}\StringTok{ }\KeywordTok{matrix}\NormalTok{(}\KeywordTok{c}\NormalTok{(}\DecValTok{1}\NormalTok{,}\DecValTok{2}\NormalTok{,}\OperatorTok{{-}}\DecValTok{1}\NormalTok{,}\DecValTok{1}\NormalTok{,}\OperatorTok{{-}}\DecValTok{1}\NormalTok{,}\OperatorTok{{-}}\DecValTok{2}\NormalTok{,}\DecValTok{3}\NormalTok{,}\DecValTok{5}\NormalTok{,}\DecValTok{4}\NormalTok{),}\DecValTok{3}\NormalTok{,}\DecValTok{3}\NormalTok{)}
\NormalTok{assign\_constraints \textless{}{-}}\StringTok{ }\KeywordTok{matrix}\NormalTok{(}\KeywordTok{c}\NormalTok{(}\DecValTok{1}\NormalTok{,}\DecValTok{2}\NormalTok{,}\DecValTok{6}\NormalTok{), }\DecValTok{3}\NormalTok{, }\DecValTok{1}\NormalTok{)}


\KeywordTok{print}\NormalTok{(}\KeywordTok{getSolutions}\NormalTok{(assign\_matrix,assign\_constraints))}
\end{Highlighting}
\end{Shaded}

\begin{verbatim}
## [1] -1.55 -0.32  0.95
## [1] -1.55 -0.32  0.95
\end{verbatim}

\end{document}
