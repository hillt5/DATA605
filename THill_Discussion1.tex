% Options for packages loaded elsewhere
\PassOptionsToPackage{unicode}{hyperref}
\PassOptionsToPackage{hyphens}{url}
%
\documentclass[
]{article}
\usepackage{lmodern}
\usepackage{amssymb,amsmath}
\usepackage{ifxetex,ifluatex}
\ifnum 0\ifxetex 1\fi\ifluatex 1\fi=0 % if pdftex
  \usepackage[T1]{fontenc}
  \usepackage[utf8]{inputenc}
  \usepackage{textcomp} % provide euro and other symbols
\else % if luatex or xetex
  \usepackage{unicode-math}
  \defaultfontfeatures{Scale=MatchLowercase}
  \defaultfontfeatures[\rmfamily]{Ligatures=TeX,Scale=1}
\fi
% Use upquote if available, for straight quotes in verbatim environments
\IfFileExists{upquote.sty}{\usepackage{upquote}}{}
\IfFileExists{microtype.sty}{% use microtype if available
  \usepackage[]{microtype}
  \UseMicrotypeSet[protrusion]{basicmath} % disable protrusion for tt fonts
}{}
\makeatletter
\@ifundefined{KOMAClassName}{% if non-KOMA class
  \IfFileExists{parskip.sty}{%
    \usepackage{parskip}
  }{% else
    \setlength{\parindent}{0pt}
    \setlength{\parskip}{6pt plus 2pt minus 1pt}}
}{% if KOMA class
  \KOMAoptions{parskip=half}}
\makeatother
\usepackage{xcolor}
\IfFileExists{xurl.sty}{\usepackage{xurl}}{} % add URL line breaks if available
\IfFileExists{bookmark.sty}{\usepackage{bookmark}}{\usepackage{hyperref}}
\hypersetup{
  pdftitle={Discussion 1},
  pdfauthor={Thomas Hill},
  hidelinks,
  pdfcreator={LaTeX via pandoc}}
\urlstyle{same} % disable monospaced font for URLs
\usepackage[margin=1in]{geometry}
\usepackage{graphicx}
\makeatletter
\def\maxwidth{\ifdim\Gin@nat@width>\linewidth\linewidth\else\Gin@nat@width\fi}
\def\maxheight{\ifdim\Gin@nat@height>\textheight\textheight\else\Gin@nat@height\fi}
\makeatother
% Scale images if necessary, so that they will not overflow the page
% margins by default, and it is still possible to overwrite the defaults
% using explicit options in \includegraphics[width, height, ...]{}
\setkeys{Gin}{width=\maxwidth,height=\maxheight,keepaspectratio}
% Set default figure placement to htbp
\makeatletter
\def\fps@figure{htbp}
\makeatother
\setlength{\emergencystretch}{3em} % prevent overfull lines
\providecommand{\tightlist}{%
  \setlength{\itemsep}{0pt}\setlength{\parskip}{0pt}}
\setcounter{secnumdepth}{-\maxdimen} % remove section numbering
\ifluatex
  \usepackage{selnolig}  % disable illegal ligatures
\fi

\title{Discussion 1}
\author{Thomas Hill}
\date{}

\begin{document}
\maketitle

\hypertarget{t10}{%
\subsection{\texorpdfstring{\textbf{T10}}{T10}}\label{t10}}

\hypertarget{an-inconsistent-system-may-have-r-n.-if-we-try-incorrectly-to-apply-theorem-fvcs-to-such-a-system-how-many-free-variables-would-we-discover}{%
\subsection{An inconsistent system may have r \textgreater{} n.~If we
try (incorrectly!) to apply Theorem FVCS to such a system, how many free
variables would we
discover?}\label{an-inconsistent-system-may-have-r-n.-if-we-try-incorrectly-to-apply-theorem-fvcs-to-such-a-system-how-many-free-variables-would-we-discover}}

An inconsistent system is one where there is no solution. This means
that there are neither one solution, nor many solutions. This is because
two of the rows contain identical coefficents but different constraints.

As an example with algebra, one equation may be \emph{x + 2y + z = 0},
and another equation \emph{x + 2y + z = -8}. Eliminating one either of
these with one another will generate a row of the matrix where there are
still is one nonzero value, but it occupies the constraint column. The
corresponding solved\\
equation would be something like \emph{0 = -8}, or if each side divided
by -8, \emph{0 = 1}. In this case, the value of r would artifically be
increaesd by 2. Graphically, this means that the equations run parallel
to each other. Omitting one equation will provide at least one soluton
but no one point can satisfy both equations as they do not intersect
with each other.

The FVCS theorem is as follows:

\hypertarget{theorem-fvcs-free-variables-for-consistent-systems}{%
\subsubsection{Theorem FVCS Free Variables for Consistent
Systems}\label{theorem-fvcs-free-variables-for-consistent-systems}}

Suppose A is the augmented matrix of a consistent system of linear
equations with n variables. Suppose also that B is a row-equivalent
matrix in reduced row-echelon form with r rows that are not completely
zeros. Then the solution set can be described with n -􀀀 r free
variables.

In this case, the value r will overcount the number of echelon-reduced
rows by the number of inconsistent equations. If r \textgreater{} n,
then the formula would result in a negative amount of free variables and
it could immediatley be concluded that this was an inconsistent system.

\end{document}
